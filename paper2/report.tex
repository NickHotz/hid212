\documentclass[sigconf]{acmart}

\usepackage{graphicx}
\usepackage{hyperref}
\usepackage{todonotes}

\usepackage{endfloat}
\renewcommand{\efloatseparator}{\mbox{}} % no new page between figures

\usepackage{booktabs} % For formal tables

\settopmatter{printacmref=false} % Removes citation information below abstract
\renewcommand\footnotetextcopyrightpermission[1]{} % removes footnote with conference information in first column
\pagestyle{plain} % removes running headers

\newcommand{\TODO}[1]{\todo[inline]{#1}}

\begin{document}
\title{Can Blockchain mitigate the Opioid Crisis through more secure Drug Distribution?}


\author{Saurabh Kumar}
\affiliation{%
  \institution{Indiana University}
  \city{Bloomington} 
  \state{IN} 
  \postcode{47408}
  \country{USA}}
\email{kumarsau@iu.edu}

\author{Mathew Schwartzer}
\affiliation{%
  \institution{Indiana University}
  \city{Bloomington} 
  \state{IN} 
  \postcode{47408}
  \country{USA}}
\email{mabschwa@iu.edu}

\author{Nicholas J Hotz}
\affiliation{%
  \institution{Indiana University}
  \city{Bloomington} 
  \state{IN} 
  \postcode{47408}
  \country{USA}}
\email{nhotz@iu.edu}


% The default list of authors is too long for headers}
\renewcommand{\shortauthors}{G. v. Laszewski}


\begin{abstract}
Like TCP/IP in the early 1980s, blockchain is a new, intriguing but grossly misunderstood technology that is still at its infancy. It is famously known as starting with Bitcoin. However, blockchain's use cases extend beyond just financial transactions and cryptocurrencies and have the potential to transform healthcare and supply chain. As the technology matures, additional transformative use cases could expand into drug distribution and theoretically could help mitigate the opioid crisis through more secure distribution of opioids.  
\end{abstract}

\keywords{HID 210, HID 212, HID225, i523, blockchain, opioid epidemic}


\maketitle



\section{Introduction}
\subsection{The Blockchain Potential}
Contracts, transactional records, and verification systems are among the foundational core of the global foundational economy. However, as Iansiti and Lakhani \cite{hbr} explain, these tools have not modernized to keep up with the needs of the rapidly evolving global economy and are ``like a rush-hour gridlock trapping a Formula 1 car.''

Blockchain, ``an open, distributed ledger that can record transactions between two parties efficiently and in a verifiable and permanent way'' \cite{hbr}, can transform various foundational pillars of the global economy by overcoming many of the antiquated shortcoming of the traditional means of managing and verifying contracts and transactions. Don Tapscott and Alex Tapscott \cite{tapscott} in their book \textit{Blockchain Revolution} propose seven design principles that, when appropriately applied, can upend existing processes and lay the foundation for a more honest, cost-effective, and accountable economy:

\begin{itemize}
\item \textit{Networked integrity:} On the blockchain, ``Participants can exchange value directly with the expectation that the other party will act with integrity.'' 
\item \textit{Distributed Power:} Since the blockchain is distributed across a broad network, it cannot be dismantled by authoritarian power, hackers, or other bad actors.
\item \textit{Value as Incentive:} Blockchains can align incentives with individual participants with the interests of the entire blockchain. This eliminates agency problems and conflicts of interests.
\item \textit{Security:} Blockchains can be secure to protect against hackers, malware, ransomware, and identity theft. Public key infrastructures, such as proposed by Satoshi for Bitcoin, verify participant activities and prevent bad actors from overriding the network. 
\item \textit{Privacy:} Blockchains provide participants with the freedom to expose a little or as much information about themselves as they desire. This allows a participant to act anonymously when desired or to share sensitive information when needed.
\item \textit{Rights Preserved:} To protect against counterfeit items, a blockchain could be established that serves a public ledger of ownership
\item \textit{Inclusion:} Access to traditional services like banking and legal counsel are limited to those who are a part of the global economy. Since the costs to entry are significantly lower, even a poor rural former on a remote corner of Earth who isn’t credit-worthy or even have access to a bank, could participate in the blockchain.
\end{itemize}

\subsection{Rational Exuberance}
Blockchain has been proposed as the ``cure'' to solve many of the world’s problems. While it indeed has the potential to help a wide variety of the world’s problem, it should not be viewed as a panacea. In order to be effective, blockchain needs to overcome numerous hurdles including its status as a nascent technology, uncertain regulatory status, large energy and computing power consumption, control, security, privacy, cultural adoption, and high initial capital costs (Rabah, 2016). Society would have to dismantle many technological, governance, organizational, and cultural barriers to create new foundations for a new world economy that relies heavily on blockchain \cite{hbr}. This will come at the cost of some existing societal norms, core business functions, and people’s jobs \cite{hbr} \cite{rabah2017overview}.

\subsection{Technology Adoption Lifecycle}
Iansiti and Lakhani \cite{hbr} argue that the process for mass adoption of blockchain may take longer than expected but will follow a fairly predictable technology adoption pattern that parallels the adoption of the blockchain to that of TCP/IP (transmission control protocol / internet protocol) whose use cases started as single-use and matured to localized uses, substitutions, and transformations. TCP/IP was introduced as a single-use in 1972 for e-mail in ARPAnet, a precursor to commercial internet for the US Department of Defense. Met with skepticism, this technology slowly gained traction among some firms in the 1980s and early 1990s for localized use and did not become mainstream until the emergence of World Wide Web in the mid 1990s. This then paved the road for infrastructure companies to provide the necessary hardware and software to establish ``plumbing'' systems for the internet. Once established, this allowed for companies to then develop businesses that substituted existing services with online services (such as Amazon books instead of Borders). Finally, a wave of companies created transformative applications that fundamentally changed service experiences (such as Napster and the music industry or Skype and telecommunications). 

Similarly, blockchain was also launched for a single use in 2008 for Bitcoin, a virtual currency. Blockchain has matured to extend beyond cryptocurrencies and is now being applied for various localized uses including in healthcare and supply chain. It took over 30 years for TCP/IP to realize its potential and it will likewise take decades for blockchain to mature into a revolutionary economic force. However, companies can start planning for this revolution today \cite{hbr}.

\section{Blockchain Applications in Supply Chain}
Blockchain, being a public ledger, can be used in different domains with slight variation in its core attributes. While the general implementation says that the data of a single block is public to all the nodes, different sets of access rights could be provided to different classes of users.  Such implementation of blockchain could be applied to the supply chain network. 

\subsection{Supply Chain Requirement}
A supply chain requires the involvement of various parties helping each other. This is generally a one to one chain network. Often, each agency uses different technologies for record keeping. Record keeping could involve any information ranging from direct communications to logistics. Trust is an important issue between agencies. Most of the agencies in a supply chain keep individual records, which are not public to other agencies in the supply chain. Agencies share some information like contracts or notarized data. An efficient management of such shared data can be accomplished using a blockchain. The blockchain provides the ability to collect, record and notarize different types of shared data\cite{arbc1}. 

\subsection{Features Provided by Blockchain}
Blockchain could also facilitate storing and maintaining logistics data. Such an application could be useful in the field of health care, where the government wants to monitor the supply of drugs. An ideal scenario for this would be to mitigate issues like the opioid crisis. Blockchain technology could simplify storage and management of trusted information. It could provide easy access of such critical public sector information to the government agencies, while providing data security\cite{arbc2}. Blocks comprise of the data records. When these blocks are added to the chain, they become immutable. This means they cannot be deleted or changed by a single agency\cite{arbc2}. A consensus has to be reached by majority of the agencies for changing any record. Such a feature helps to maintain the security of the records by eliminating data corruption. Each block is verified and managed using some shared protocols. This process can be automated to allow ease of data entry.

\subsubsection{Trust}
In a supply chain no single agency is trusted to maintain the records. While all agencies must approve the contents of the record in order to avoid disputes. Therefore, records should have a time stamp and an origin proof. Normally, a third party facilitates this requirement. Blockchains can provide an alternate solution, where agencies jointly manage the records and preventing corruption by a single agency\cite{arbc1}. 

\subsubsection{Access}
Blockchain technology enforces identical data to be stored by each agency. When one copy is updated, all the other copies are also updated. This eliminates the need of a third party to facilitate management of records\cite{arbc3}. Alternatively, different levels of read and write access could be provided to different agency. Although some meta data should be stored in the public ledger. 

\subsubsection{Redundancy and Security}
Blockchain also assist in providing security by disallowing redundancy at the same node. In the areas of logistics and inventory data, blockchain provides a new approach to supply chain management. The core logic of blockchain does not allow duplicate entries to be created in the same place\cite{arbc4}. An unique inventory can have a single entry with multiple updates, but not duplication. This prevents the agencies from creating fake date. In the example of a drug inventory, the shipment status for a batch of drugs will be updated for everyone, everywhere. Each entry could be tracked back to its origin\cite{arbc4}. 

\subsubsection{Transparency}
Transparency in a business helps to grow trust between agencies. Sharing information can improve relationships between agencies. Without blockchain, transparency is a hard goal to achieve. Blockchains can help improve visibility of contracts, legal documents as well as other inter agency data\cite{pabc1}. Agencies are not obligated to show all of their data. Some access can be provide to data that could be useful to other agencies and a shared collection of records can also be stored and managed by co-operation from different agencies.

\subsection{Concepts provided by Blockchain}

\subsubsection{Node}
They are the individual units connected to the blockchain network. They are computers with adequate software to maintain a blockchain\cite{pabc1}\cite{pabc2}. The blockchain network connects all the nodes and can read and write data to a block.

\subsubsection{Block}
They are the group of records, bundled together by nodes. They follow a specific set of rules and have limited size. Blocks are also linked to the last generates block, thus forming a chain\cite{pabc1}.

\subsubsection{Smart Contracts}
These are the codes with time stamps to represent a contract\cite{pabc1}.

\subsubsection{Submit Transaction}
In case of a new transaction submission to the network, an individual node circulates it to all the other nodes in the network\cite{pabc1}. The main purpose for circulation is approval. 

\subsubsection{Transaction Approval}
When a transaction is submitted and circulated in the network, each node verifies it. Invalid transactions are deleted\cite{pabc1}.

\subsubsection{Consensus}
For multiple systems to work in a distributed network, they must have an agreement. Such a structure is useful in case of fault tolerance when those agreed set of protocols help to restore the data\cite{pabc1}.

\subsection{Data Types in Blockchain}
There are three major types of data stored on a blockchain, namely un-encrypted, encrypted and hashed\cite{arbc1}. All the agencies have read access to the un-encrypted data. Such data is fully transparent and facilitates immediate dispute resolution. The encrypted data can only be read by the agencies with the access to such data. This means an agency should have a decryption key to read to read the encrypted data. Encrypted data provides restricted access but is also stored in every node in the blockchain. In case of a dispute the decryption key could be used by different agencies to rectify the entry or deletion of any record. Hash data is also a hidden data, where hash keys act like fingerprints to represent changes or entry for any data record. Each agency can easily confirm their hash keys. Breaking the hash key is next to impossible. Only hash key is placed in the blockchain. The record data is stored off-chain by individual agencies. Data could be revealed, in case of a dispute, by the respective agency\cite{arbc1}.

\subsection{Specialized Use Cases of Blockchain}
\subsubsection{Counterfeit Detection}
A specialized use case for blockchain in supply chain management could be the detection of counterfeits. These are the fraud duplicates of the original product that enter the supply chain. If a blockchain model is enforced on the supply chain, each batch of product can be tracked back to its origin. This means that each batch will have a block of code associated with it. If a batch does not have its presence in the blockchain, then it can be deemed as a counterfeit.  

\subsubsection{Data Analysis}
The data stored in each block can be used to analyze patterns in a product consumption and supply. Government can use this data to regulate the supply of critical products like drugs. This could also be used the keep checks on the activities of different agencies. Trend analysis can help to provide predictions and the usage and over various locations for different time of the year. This can also assist in predicting shortages and the adequate supply for a disaster, natural or artificial, situation. 

\section{The Opioid Crisis}
 “Starting in the late 1990’s, pharmaceutical companies reassured the medical community that patients would not become addicted to prescription opioid pain relievers, and healthcare providers began to prescribe them at greater rates” \cite{opsis1}. In fact, “the US population, which represents approximately 5\% of the Earth’s population, consumed approximately 80\% of the global supply of prescribed opioids in the first decade of this century” \cite{Vowles01}. Between 1999 and 2015 the amount of prescribed opioids pain killers such as codeine, fentanyl, oxycodone, Demerol, and Vicodin quadrupled. Consequently, in the same time period, opioid related deaths also quadrupled.  With no signs of stopping, this epidemic is burgeoning across America killing nearly 91 people a day\cite{opsis10}. In October 2017 President Trump was forced to declare the opioid epidemic “a national health emergency.”\cite{opsis3}. The addictive nature of prescribed opioid pain killers increases the “potential for unforeseen adverse events for the patient, including overdose, experience of physiological dependence and subsequent withdrawal, addiction, and negative impacts on functioning” (Vowels01). Patients with wholesome medical intentions often fall victim to the pills addictive nature. Misuse and eventual abuse of prescribed opioid painkillers is common, 21\%-29\% of patients prescribed opioids for chronic pain misuse them while 7.8\%-11.7\% develop an addiction \cite{Vowles01}. An opioid addiction often serves a gateway to other illegal drug use. With similar highs, prescription opioid addicts often transition to heroin since it is cheaper and easier to obtain. In fact, 4\%-6\% of patients using prescribed opioids develop a heroin addiction\cite{opsis1}. Whereas, 75\% of heroin users began their opioid addiction with prescription opioids \cite{Cicero01}. Examining the most recent demographic information of opioid deaths, it is evident that rural white American’s males are at the greatest risk of an opioid overdose. The latest data collected in 2015 shows a total of 33,091 Americans died as a result of an opioid overdose. 27,056 white Americans died, 2,741 black Americans died, and 2,507 Hispanic Americans died. People aged 0-24 are the least likely age group to die from an opioid overdose comprising 10\% of the total opioid related deaths, where 25-34 year-olds constitute 26\% of opioid deaths, 35-44 year-olds constitute 23\% of opioid deaths, 45-54 year-olds constitute 23\% of opioid deaths, and 55+ individuals constitute 19\% of opioid deaths. Males are almost twice as likely to die from an opioid overdose at 65\% compared with 35\% for females\cite{opsis4}.

\subsection{Financial Impact} 
The health impacts are the primary reason for concern, but the financial liability associated with the epidemic is also increasing. In 2007 the estimated financial impact of the crisis reached \$55.7 billion \cite{Birnbaum01}. While in 2013 the estimated financial impact grew to \$78.5 billion (Florence01). Of the total economic burden, roughly 25\% or \$20 billion is conveyed to the public sector \cite{Florence01}. Partitioned between workplace, healthcare, and criminal justice costs, the overall financial burden will continue to rise until a reversal in current trends. Finding a solution to the epidemic is especially important for opioid drug makers as lawsuits accusing pharmaceutical companies of deceptive marketing are common place. After a U.S. Justice Department probe in 2007, the maker of OxyContin pleaded guilty to federal charges and agreed to pay a total of \$634.5 million. In later cases, OxyContin maker Purdue Pharma LP settled two additional cases for a combined \$43.5 million. Since then states litigating the culpability of opioid drug makers include “South Carolina, Oklahoma, Mississippi, Ohio, Missouri and New Hampshire as well as cities and counties in California, Illinois, Ohio, Oregon, Tennessee and New York\cite{opsis11}. In a suit filed in April 2017 against the three largest drug distributors in America and drug retailers CVS, Walgreens and Walmart, lawyers for plaintiffs Cherokee Nation claim that the “Defendants turned a blind eye to the problem of opioid diversion and profited from the sale of prescription opioids to the citizens of the Cherokee Nation in quantities that far exceeded the number of prescriptions that could reasonably have been used for legitimate medical purposes”\cite{opsis5}.

\subsection{Proposed solutions}
Citizens, drug makers, and governments recognize the importance of solving this crisis. In one example, Indiana University recently committed \$50 million to fund studies that lead to a decline in opioid related deaths in Indiana. Through policy and politics, the federal government is also attempting to find solutions to the epidemic. In the same address President Trump declared the opioid epidemic a national health crisis, he proposed “really tough, really big, really great advertising” to adjust public opinion on the adverse effects of current pain killers\cite{opsis6}. In April 2017 Tom Price of the U.S. Department of Health and Human Services outlined five concurrent solutions:
\begin{itemize}
\item “Improving access to treatment and recovery services;
\item Promoting use of overdose-reversing drugs;
\item Strengthening our understanding of the epidemic through better public health surveillance;
\item Providing support for cutting edge research on pain and addiction; and
\item Advancing better practices for pain management”\cite{opsis7}.
\end{itemize} 

Similarly, researchers at the Network for Public Health Law, Boston University, and Northeastern University propose a four-step solution:
\begin{itemize}
\item “Improving Clinical Decision Making
\item Improving Access to Evidence-Based Treatment
\item Investing in Comprehensive Public Health Approaches
\item Re-focusing Law Enforcement Response \cite{Davis01}.”
\end{itemize}

\section{Pharmacutical Supply Chain}
It is evident that big data analytics and methods will be a vital component of any solution. President Trump’s Commission on Combating Drug Addiction and the Opioid Crisis repeatedly mentions ‘data sharing’ as a method to cope and limit the opioid crisis\cite{opsis3}. Data sharing is critical in the application of The Drug Quality and Security Act signed by President Barack Obama in 2013. The bill aims to make prescription drugs safer through more transparent manufacturing (Title I) and supply chains (Title II). Title II mandates include systems to trace lot-level historical transactions, systems to verify product legitimacy, company required federal licensure, and authentication of licensed trading partners. These massive changes place immense financial pressure on pharmaceutical companies, drug distributors, and prescribers to develop sustainable supply chain solutions. The 2023 deadline gives pharma companies time to test and implement the most sustainable and practical solution \cite{opsis8}. The Drug Quality and Security Act was written in response to a manufacturing contamination that lead to a deadly meningitis outbreak that killed 64 people. Although the current opioid epidemic does not prompt the same policy-making fear, the Act’s mandates will aid in reducing the crisis through a more transparent supply chain. It is estimated that 10\% of the global pharmaceutical trade is counterfeit \cite{Kelesidis01}. There are many weaknesses in the current pharmaceutical supply chain. Lead times for opioid manufacturing can take up to 300 days and within this time period, prescription drugs pass through five different value-added nodes:
\begin{itemize}
\item (1) primary manufacturing
\item (2) secondary manufacturing
\item (3) market warehouses/distribution centers
\item (4) wholesalers and
\item (5) pharmacies/hospitals \cite{Shah01}.
\end{itemize}

At every transition node, the prescription drugs encounter an outbound handler, at least one shipper, and an inbound handler. In a supply chain where a product passes through all five nodes, there are 18 different opportunities for mismanagement and disruption. Pharmaceutical supply chains are further complicated by the elasticity of their products and the need for manufacturing companies to operate a ‘push’ strategy. High pipeline and safety stocks are necessary to supply an unknown future demand. At any given time, there is usually 4–24 weeks of finished goods in the supply chain \cite{Shah01}. Researchers at The Medicine Group propose using radio frequency identification (RFID) technology to combat the high prevalence of counterfeit pharmaceuticals\cite{Taylor01}. Tracking pharmaceuticals lots through RFID technology has the potential to reduce costs, increase transparency, and identify counterfeit lots. 

\subsection{RFID Tags}
RFID technology definitely has its advantages over current barcode tracking methods. RFID tags can hold up to 32,000 alphanumeric characters compared to just 20 in a barcode. RFID tags have a much higher upfront cost but downstream they decrease total handling cost due to the timely process to scan each individual barcode. Unlike RFID tags, barcodes are susceptible to wear and tear and are easily replicated. RFID technology has its flaws, each tag cost 5-10 US cents, tags are vulnerable to electromagnetic interference and poor manufacturing, tags are larger, and generate 100 times more data\cite{Taylor01}\cite{opsis9}. From a security perspective, RFID technology is a good option to conform to Title II of The Drug Quality and Security Act.

\section{Blockchain in the Opioid Crisis}
Blockchain technology has the potential to reduce the opioid epidemic by providing a clear batch-level transnational history from raw material mining through patient level distribution. There are many advantages to using an open and decentralized supply chain that blockchain offers. For example, collaborative decision making between manufactures and distributors creates both economic and social benefits \cite{Nematollahi01}. 

Drug monitoring programs (PMPs) simulate blockchain technology. States with strong (PMPs) show a reduction in the number of opioid related deaths through identifying high-risk prescribing and patient behaviors \cite{pardo01}\cite{patrick01}. The quality of PMPs depends on the quality of their data. To identify high risk individuals, PMPs rely on supply chain transactional histories and city specific demographic and opioid usage trends. 

\section{CONCLUSION}
Although still in its infancy, blockchain has the potential to be just as transformative as TCP/IP. Early and potential applications in healthcare and supply chain suggest that blockchain is indeed moving along the path of technology adoption. Because blockchain is a low cost solution for supply chain management and provides security and transparency, it can be used for digital data and communication to overall the distribution of controlled substances such as opioids but this model has yet to be tested. But the computation and infrastructure cost for the entire model is low and should be tested to develop a proof of concept system that leverages blockchain to more securely distribute perscription opioids.

\begin{acks}

  The authors would like to thank Dr. Gregor von Laszewski for his support and suggestions to write this paper.

\end{acks}

\bibliographystyle{ACM-Reference-Format}
\bibliography{report} 

\appendix

We include an appendix with common issues that we see when students
submit papers. One particular important issue is not to use the
underscore in bibtex labels. Sharelatex allows this, but the
proceedings script we have does not allow this.

When you submit the paper you need to address each of the items in the
issues.tex file and verify that you have done them. Please do this
only at the end once you have finished writing the paper. To d this
cange TODO with DONE. However if you check something on with DONE, but
we find you actually have not executed it correcty, you will receive
point deductions. Thus it is important to do this correctly and not
just 5 minutes before the deadline. It is better to do a late
submission than doing the check in haste. 

\section{Issues}

\DONE{Example of done item: Once you fix an item, change TODO to DONE}

\subsection{Uncaught Bibliography Errors}

    \DONE{Bibtex labels cannot have any spaces, \_ or \& in it}
    \DONE{Citations in text showing as [?]: this means either your report.bib is not up-to-date or there is a spelling error in the label of the item you want to cite, either in report.bib or in report.tex}

\subsection{Formatting}

    \DONE{HID and i523 not included in the keywords}

\subsection{Writing Errors}

    \DONE{Errors in title capitalization}
    \DONE{A few spelling erros}

\subsection{Citation Issues and Plagiarism}

    \DONE{It is your responsibility to make sure no plagiarism occurs. The instructions and resources were given in the class}
    \DONE{Too few references}
    \DONE{The citation mark should not be in the beginning of the sentence or paragraph, but in the end, before the period mark. example: ... a library called Message Passing Interface(MPI) [7].}
    \DONE{Put a space between the citation mark and the previous word}

\subsection{Character Errors}

    \DONE{Pasting and copying from the Web often results in non-ASCII characters to be used in your text, please remove them and replace accordingly. This is the case for quotes, dashes and all the other special characters.}

\end{document}
