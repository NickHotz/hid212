\documentclass[sigconf]{acmart}

\usepackage{graphicx}
\usepackage{hyperref}
\usepackage{todonotes}

\usepackage{endfloat}
\renewcommand{\efloatseparator}{\mbox{}} % no new page between figures

\usepackage{booktabs} % For formal tables

\settopmatter{printacmref=false} % Removes citation information below abstract
\renewcommand\footnotetextcopyrightpermission[1]{} % removes footnote with conference information in first column
\pagestyle{plain} % removes running headers

\newcommand{\TODO}[1]{\todo[inline]{#1}}

\begin{document}
\title{Can Blockchain Mitigate the Opioid Crisis Through More Secure Drug Distribution?}


\author{Saurabh Kumar}
\affiliation{%
  \institution{Indiana University}
  \city{Bloomington} 
  \state{IN} 
  \postcode{47408}
  \country{USA}}
\email{kumarsau@iu.edu}

\author{Mathew Schwartzer}
\affiliation{%
  \institution{Indiana University}
  \city{Bloomington} 
  \state{IN} 
  \postcode{47408}
  \country{USA}}
\email{mabschwa@iu.edu}

\author{Nicholas J Hotz}
\affiliation{%
  \institution{Indiana University}
  \city{Bloomington} 
  \state{IN} 
  \postcode{47408}
  \country{USA}}
\email{nhotz@iu.edu}

\begin{abstract}
Like TCP/IP in the 1970s and 1980s, blockchain is a new, intriguing but grossly misunderstood technology that is still in its infancy that is famously known as starting with Bitcoin. However, blockchain's use cases extend beyond just financial transactions and cryptocurrencies and have the potential to transform nearly every industry including healthcare and supply chain. As the technology matures, additional transformative use cases could expand into drug distribution; specifically, a blockchain prescription opioid distribution system could theoretically mitigate some aspects of the opioid crisis.  
\end{abstract}

\keywords{HID 210, HID 212, HID 225, i523, Blockchain, Pharmaceutical Supply Chain, Opioid Epidemic}

\maketitle

\section{Introduction}
\subsection{The Need to Modernize Global Record Keeping}
Contracts, transactional records, and verification systems are part of the foundational core of the global economy. However, as Iansiti and Lakhani \cite{hbr} explain, these tools have not modernized to keep up with the needs of the rapidly evolving global economy and are ``like a rush-hour gridlock trapping a Formula 1 car.'' Record management is still being maintained the same way as it was in the 20th century which creates broad consequences for nearly every industry including supply chain and healthcare. SAURABH--2-4 lines on why modern supply chain sucks. In the USA in 2014 healthcare fraud cost an estimated \$272 billion \cite{economist2014}, and in 2016, healthcare data breaches impacted over 27 million patients \cite{das2017}. Today, medical data management is stifled by antiquated technology that limits patients' ability to manage and control access to their electronic medical records \cite{ekblaw2016medrec}.  MATT--2-4 LINES that are like and drug distributions is fraught with counterfeits and duplicate drug distributions and are lost in transit, stolen, etc...These are just a few of the issues that suggest that record management should upgrade onto the blockchain.

\subsection{Rational Exuberance for Blockchain}
Blockchain, ``an open, distributed ledger that can record transactions between two parties efficiently and in a verifiable and permanent way'' \cite{hbr}, has the potential to resolve these and other fundamental problems of the global economy by overcoming many of the antiquated shortcomings of the traditional means of managing and verifying contracts and transactions. However, like TCP/IP in the 1970s and 1980s, blockchain is an immature technology that faces numerous challenges to mass adoption. In spite of its current limitations, blockchain is already seeing promising applications in various industries extending beyond just finance including healthcare and supply chain. One particularly exciting use case sits at the intersection of healthcare and supply chain to develop a more secure distribution system for opioid medications that could potentially mitigate the opioid crisis.

\section{Blockchain Overview}
\subsection{The Blockchain Framework}
Blockchain is a foundational technology that uses numerous technological pieces. Some of the most significant pieces follow.

\subsubsection{Node} Nodes are the individual units connected to the blockchain network. They are computers with adequate software to maintain a blockchain \cite{pabc1} \cite{pabc2}. The blockchain network connects all the nodes and can read and write data to a block.

\subsubsection{Block} Blocks are the group of records, bundled together by nodes. They follow a specific set of rules and have limited size. Blocks are also linked to the last generated block, thus forming a chain \cite{pabc1}.

\subsubsection{Smart Contracts} Smart contracts are the codes with time stamps to represent a contract \cite{pabc1}. Iansiti and Lakhani \cite{hbr} believe that ``'smart contracts' may be the most transformative blockchain application at the moment,'' because they allow for automatic payments whenever contract conditions are met. 

\subsubsection{Submit Transaction} In case of a new transaction submission to the network, an individual node circulates it to all the other nodes in the network \cite{pabc1}. The main purpose of circulation is approval. 

\subsubsection{Transaction Approval} When a transaction is submitted and circulated in the network, each node verifies it. Invalid transactions are deleted \cite{pabc1}.

\subsubsection{Consensus} For multiple systems to work in a distributed network, they must have an agreement. Such a structure is useful in case of fault tolerance when those agreed set of protocols help to restore the data \cite{pabc1}.

\subsection{Data Types in Blockchain}
There are three major types of data stored on a blockchain, namely un-encrypted, encrypted and hashed \cite{arbc1}. 

\subsubsection{Un-encrypted Data} All the agencies have read access to the un-encrypted data. Such data is fully transparent and facilitates immediate dispute resolution.

\subsubsection{Encrypted Data} The encrypted data can only be read by the agencies with the access to such data. This means an agency should have a decryption key to read to read the encrypted data. Encrypted data provides restricted access but is also stored in every node in the blockchain. In case of a dispute, the decryption key could be used by different agencies to rectify the entry or deletion of any record.

\subsubsection{Hash Data} Hash data is also a hidden data, where hash keys act like fingerprints to represent changes or entry for any data record. Each agency can easily confirm their hash keys. Breaking the hash key is next to impossible. Only the hash key is placed in the blockchain. The record data is stored off-chain by individual agencies. Data could be revealed, in case of a dispute, by the respective agency \cite{arbc1}.

\subsection{Benefits of Blockchain}
Blockchain's framework and data types provide such broad-ranging benefits that blockchain has been proposed as the ``cure'' to solve many of the world's problems. This exuberance stems from the fundamental benefits that are cornerstones to nearly every industry.

\subsubsection{Trust} In a supply chain no single agency is trusted to maintain the records. Instead, all agencies must approve the contents of the record in order to avoid disputes. Therefore, records should have a time stamp and an origin proof. Normally, a third party facilitates this requirement. Blockchains can provide an alternate solution, where agencies jointly manage the records and preventing corruption by a single agency \cite{arbc1}. 

\subsubsection{Access} Blockchain technology enforces identical data to be stored by each agency. When one copy is updated, all the other copies are also updated. This eliminates the need for a third party to facilitate management of records \cite{arbc3}. Alternatively, different levels of read and write access could be provided to different agencies. Although some meta data should be stored in the public ledger. 

\subsubsection{Redundancy and Security} Blockchain also assists in providing security by disallowing redundancy at the same node. In the areas of logistics and inventory data, blockchain provides a new approach to supply chain management. The core logic of blockchain does not allow duplicate entries to be created in the same place \cite{arbc4}. A unique inventory can have a single entry with multiple updates, but not duplication. This prevents the agencies from creating false information. In the example of a drug inventory, the shipment status for a batch of drugs will be updated for everyone, everywhere. Each entry could be tracked back to its origin \cite{arbc4}. 

\subsubsection{Transparency} Transparency in a business helps to grow trust among agencies. Sharing information can improve relationships among these agencies. Without blockchain, transparency is hard to achieve. Blockchains can help improve the visibility of contracts, legal documents as well as other inter-agency data \cite{pabc1}. Agencies are not obligated to show all of their data. Some access can be provided to data that could be useful to other agencies and a shared collection of records can also be stored and managed by co-operation from different agencies.

\subsubsection{Low Transaction Costs} Through by-passing third-party verification systems such as brokers, lawyers, or banks, blockchain could significantly reduce transaction costs. Not only will this lower costs for existing transactions, it could open up the market for micro-payments \cite{hbr}. 

\subsection{Challenges to Blockchain Mass Adoption}
While it indeed has the potential to help a wide variety of the world's problems, it should not be viewed as a panacea.  Blockchain is not mature enough to support mass-market adoption and faces numerous challenges. Rabah \cite{rabah2017overview} states that in order to be effective, blockchain needs to overcome its status as a nascent technology, uncertain regulatory status, large energy and computing power consumption, control, security, privacy, cultural adoption, and high initial capital costs. Tapscott and Tapscott \cite{tapscott} agree that its current technical infrastructure is not sufficient, its energy consumption and computational requirements are not sustainable, and user-friendly systems have yet to be adopted that would allow for mass market adoption.

Society would have to dismantle many technological, governance, organizational, and cultural barriers to create new foundations for a new world economy that relies heavily on blockchain \cite{hbr}. This will come at the cost of some existing societal norms, core business functions, and people's jobs \cite{hbr} \cite{rabah2017overview}. 

\subsection{Technology Adoption Lifecycle}
Iansiti and Lakhani \cite{hbr} argue that the process for mass adoption of blockchain may take longer than expected but will follow a fairly predictable technology adoption pattern that parallels the adoption of TCP/IP (transmission control protocol / internet protocol). TCP/IP started as \textit{single-use} and matured to \textit{localized uses}, \textit{substitutions}, and \textit{transformations}. It was introduced as a \textit{single-use} in 1972 for e-mail in ARPAnet, a precursor to commercial internet for the US Department of Defense. Met with skepticism, this technology slowly gained traction among some firms in the 1980s and early 1990s for \textit{localized use} and did not become mainstream until the emergence of World Wide Web in the mid-1990s. This then paved the road for infrastructure companies to provide the necessary hardware and software to establish ``plumbing'' systems for the internet. Once the technical infrastructure was mature enough, companies then developed businesses that \textit{substituted} existing services with online services (such as Amazon books instead of Borders). Finally, a wave of companies created \textit{transformative} applications that fundamentally changed service experiences (such as Napster in the music industry or Skype in telecommunications).

Similarly, blockchain was also launched for a \textit{single use} in 2009 for Bitcoin, a virtual currency. Blockchain has matured to extend beyond cryptocurrencies and is now being applied for various \textit{localized uses} including in healthcare and supply chain. It took over 30 years for TCP/IP to realize its potential and it will likewise take decades for blockchain to mature into a revolutionary economic force. However, companies can start planning for this revolution today and implement blockchains that follow seven design principles \cite{hbr} \cite{tapscott}.

\subsection{Seven Design Principles for Blockchain}
Tapscott and Tapscott \cite{tapscott} in their book \textit{Blockchain Revolution} propose seven design principles that, when appropriately applied, can help blockchain move down the technology adoption lifecycle and create more honest, cost-effective, and accountable economy systems.

\subsubsection{Networked integrity} Because all agencies on the blockchain must approve updates, ``Participants can exchange value directly with the expectation that the other party will act with integrity.'' \cite{tapscott}. 

\subsubsection{Distributed Power} Since the blockchain is distributed across a broad network, it cannot be dismantled by authoritarian power, hackers, or other bad actors. There are no single points of failure and the blockchain can still perpetuate even if numerous nodes are compromised \cite{tapscott}.

\subsubsection{Value as Incentive} Blockchains can align incentives of individual participants with the interests of the entire blockchain. This minimizes agency problems and conflicts of interests \cite{tapscott}.

\subsubsection{Security} Blockchains can protect against hackers, malware, ransomware and identity theft by using a variety of security features. Public key infrastructures, private keys, public keys, and verification methods verify participant activities and prevent bad actors from overriding the network \cite{tapscott}. 

\subsubsection{Privacy} Blockchains can and should provide participants with the freedom to expose as little or as much information about themselves as they desire. This allows a participant to act anonymously when desired or to share sensitive information with only appropriate parties when needed \cite{tapscott}.

\subsubsection{Rights Preserved} To protect against counterfeit items, a blockchain can serve as a public ledger of ownership \cite{tapscott}.

\subsubsection{Inclusion} Currently, access to certain financial services is limited to those who are deemed ``creditworthy''. Blockchains can and should have significantly lower bars of entry that are not managed by banking institutions so that even a poor rural former on a remote corner of Earth who isn't creditworthy, could participate in the blockchain \cite{tapscott}.

\section{Blockchain Applications}
\subsection{Supply Chain}
Blockchain, being a public ledger, can be used in different domains with slight variation in its core attributes. While the general implementation says that the data of a single block is public to all the nodes, different sets of access rights could be provided to different classes of users.  Such implementation of blockchain could be applied to a supply chain network. 

A supply chain requires the involvement of various parties helping each other. This is generally a one-to-one chain network. Often, each agency uses different technologies for record keeping. Record keeping could involve any information ranging from direct communications to logistics. Trust is an important issue between agencies. Most of the agencies in a supply chain keep individual records, which are not public to other agencies in the supply chain. Agencies share some information like contracts or notarized data. An efficient management of such shared data can be accomplished using a blockchain. The blockchain provides the ability to collect, record, and notarize different types of shared data \cite{arbc1}. 

Blockchain could also facilitate storing and maintaining logistics data. Such an application could be useful in the field of healthcare, where the government wants to monitor the supply of drugs. An ideal scenario for this would be to mitigate issues like the opioid crisis. Blockchain technology could simplify storage and management of trusted information. It could provide easy access of such critical public sector information to government agencies while providing data security \cite{arbc2}. Blocks comprise of the data records. When these blocks are added to the chain, they become immutable. This means they cannot be deleted or changed by a single agency \cite{arbc2}. A consensus has to be reached by a majority of the agencies for changing any record. Such a feature helps to maintain the security of the records by eliminating data corruption. Each block is verified and managed using some shared protocols. This process can be automated to allow ease of data entry. Two uses cases are for counterfeit detection and data analysis.

\subsubsection{Counterfeit Detection} A specialized use case for blockchain in supply chain management could be the detection of counterfeits which are fraud duplicates of the original product that enter the supply chain. If a blockchain model is enforced on the supply chain, each batch of product can be tracked back to its origin. This means that each batch will have a block of code associated with it. If a batch does not have its presence in the blockchain, then it can be deemed as a counterfeit.

\subsubsection{Data Analysis} The data stored in each block can be used to analyze patterns in a product consumption and supply. Governments can use this data to regulate the supply of critical products like drugs. This could also be used the keep checks on the activities of different agencies. Trend analysis can help to provide predictions and the usage and over various locations for different time of the year. This can also assist in predicting shortages and the adequate supply for a disaster, natural or artificial, situation. 

\subsection{Healthcare}
Representing over 17\% of the United States' GDP, healthcare costs continue to soar \cite{hitchingHealthcare}. More effective data management could address many of healthcare's fundamental issues, and according to a 2011 McKinsey report \cite{mckinsey2011}, more effective health data management could save \$300 billion annually. Current innovations focus on placing patients at the center, privacy and access, completeness of information, and cost \cite{hitchingHealthcare}. Three interesting applications of blockchain for healthcare are in claims adjudication, cyber security and healthcare IoT, and electronic medical records. Three uses cases are claims adjudication and fraud prevention, cyber security and healthcare IoT, and electronic medical records.

\subsubsection{Claims Adjudication and Fraud Prevention}
In 2014, the Economist estimated that the United States wastes \$272 billion dollars on healthcare fraud \cite{economist2014}. Blockchain could not only minimize fraudulent billing; but, by automating claims adjudication and billing processes, it obviates the need for administrative and transactional costs through third parties. Gem Health and Capital One are developing a blockchain-based solution for healthcare claims management \cite{das2017}.

\subsubsection{Cyber Security and Healthcare IoT}
In 2016, there were 450 reported health data breaches, impacting 27 million patients. Hacking and ransomware were responsible for 27\% of these breaches. Each additional connected medical device serves as a potential entry point for bad actors. With an estimated 20-30 billion healthcare IoT devices by 2020, blockchain could secure these devices and protect confidential data. Telstra, IBM, and Tierion are three companies that are developing cyber security solutions for connected healthcare devices \cite{das2017}.

\subsubsection{Electronic Medical Records} Beleaguered by stifled technology development, limited ownership control by patients, fragmented information systems, and risks of electronic protected health information hacking, electronic medical records have perhaps the most important use cases for blockchain \cite{yuan2016blockchains}. Blockchain can provide interoperability of healthcare information, improved security, patient-centric control, and immutable records \cite{das2017}. Three examples of blockchain-based EMRs include MedRec, Medicalchain, and the Estonian eHealth Foundation. First, by leveraging smart contracts on the Ethereum blockchain, MedRec is a prototype system that provides patients with ``one-stop-shop access to their medical history'' and shows promise to give ownership of health information back to the patients who can selectively share access through a modern API interface in a secure manner \cite{ekblaw2016medrec}. Second, Medicalchain is a permissioned blockchain distributed on networks of international healthcare providers that allow patients to transfer medical records across national borders \cite{hitchingHealthcare}. Third, a data security company called Guardtime is using its Keyless Signature Infrastructure system in partnership with the Estonian eHealth Foundation to store Estonian health records on a blockchain.

\section{The Opioid Crisis}
 ``Starting in the late 1990s, pharmaceutical companies reassured the medical community that patients would not become addicted to prescription opioid pain relievers, and healthcare providers began to prescribe them at greater rates'' \cite{opsis1}. In fact, ``the US population, which represents approximately 5\% of the Earth's population, consumed approximately 80\% of the global supply of prescribed opioids in the first decade of this century '' \cite{Vowles01}. Between 1999 and 2015 the amount of prescribed opioids painkillers such as codeine, fentanyl, oxycodone, Demerol, and Vicodin quadrupled. Consequently, in the same time period, opioid-related deaths also quadrupled.  With no signs of stopping, this epidemic is burgeoning across America killing nearly 91 people a day \cite{opsis10}. 
 
 \subsection{Health Impact}
 In October 2017 President Trump was forced to declare the opioid epidemic ``a national health emergency'' \cite{opsis3}. The addictive nature of prescribed opioid painkillers increases the``potential for unforeseen adverse events for the patient, including overdose, experience of physiological dependence and subsequent withdrawal, addiction, and negative impacts on functioning'' \cite{Vowles01}. 
 
 Patients with wholesome medical intentions often fall victim to the pills' addictive nature. Misuse and eventual abuse of prescribed opioid painkillers are common: 21\%-29\% of patients prescribed opioids for chronic pain misuse them while 7.8\%-11.7\% develop an addiction \cite{Vowles01}. Moreover, an opioid addiction often serves as a gateway to other illegal drug use. With similar highs, prescription opioid addicts often transition to heroin, an illicit street-made opioid, since it is cheaper and easier to obtain. In fact, 4\%-6\% of patients using prescribed opioids develop a heroin addiction \cite{opsis1}. Whereas, 75\% of heroin users began their opioid addiction with prescription opioids \cite{Cicero01}.
 
 Examining the most recent demographic information of opioid deaths, it is evident that rural white males are at the greatest risk of an opioid overdose. The latest data collected in 2015 shows a total of 33,091 Americans died as a result of an opioid overdose. 27,056 white Americans died, 2,741 black Americans died, and 2,507 Hispanic Americans died. People aged 0-24 are the least likely age group to die from an opioid overdose comprising 10\% of the total percentage of opioid-related deaths, where 25-34 year-olds constitute 26\% of opioid deaths, 35-44 year-olds constitute 23\% of opioid deaths, 45-54 year-olds constitute 23\% of opioid deaths, and 55+ individuals constitute 19\% of opioid deaths. Males are almost twice as likely to die from an opioid overdose at 65\% compared with 35\% for females \cite{opsis4}.

\subsection{Financial Impact} 
The health impacts are the primary reason for concern, but the financial liability associated with the epidemic is also increasing. In 2007, the estimated financial impact of the crisis reached \$55.7 billion \cite{Birnbaum01}. While in 2013 the estimated financial impact grew to \$78.5 billion \cite{Florence01}. Of the total economic burden, roughly 25\% or \$20 billion is conveyed to the public sector \cite{Florence01}. Partitioned between workplace, healthcare, and criminal justice costs, the overall financial burden will continue to rise until a reversal in current trends. 

Finding a solution to the epidemic is especially important for opioid drug makers as lawsuits accusing pharmaceutical companies of deceptive marketing are commonplace. After a U.S. Justice Department probe in 2007, the maker of OxyContin pleaded guilty to federal charges and agreed to pay a total of \$634.5 million. In later cases, OxyContin maker Purdue Pharma LP settled two additional cases for a combined \$43.5 million. Since then states litigating the culpability of opioid drug makers include ``South Carolina, Oklahoma, Mississippi, Ohio, Missouri and New Hampshire as well as cities and counties in California, Illinois, Ohio, Oregon, Tennessee and New York'' \cite{opsis11}. In a suit filed in April 2017 against the three largest drug distributors in America and drug retailers CVS, Walgreens and Walmart, lawyers for plaintiffs Cherokee Nation claim that the ``Defendants turned a blind eye to the problem of opioid diversion and profited from the sale of prescription opioids to the citizens of the Cherokee Nation in quantities that far exceeded the number of prescriptions that could reasonably have been used for legitimate medical purposes'' \cite{opsis5}.

\subsection{Proposed Solutions}
Citizens, drug makers, and governments recognize the importance of solving this crisis. In one example, Indiana University recently committed \$50 million to fund studies that lead to a decline in opioid-related deaths. Through policy and politics, the federal government is also attempting to find solutions to the epidemic. In the same address President Trump declared the opioid epidemic a national health crisis, he proposed ``really tough, really big, really great advertising'' \cite{opsis6}. Tom Price of the U.S. Department of Health and Human Services outlined a more detailed federal plan including, ``improving access to treatment and recovery services, promoting use of overdose-reversing drugs, strengthening our understanding of the epidemic through better public health surveillance, providing support for cutting edge research on pain and addiction, and advancing better practices for pain management''\cite{opsis7}. Cities are conflicted on using...

In a similar proposal, researchers at the Network for Public Health Law, Boston University, and Northeastern University proposed a four-step solution including ``improving clinical decision making and access to evidence-based treatment, investing in comprehensive public health approaches, and re-focusing law enforcement response ''\cite{Davis01}. 

Current research shows that states with strong drug prescription monitoring programs (PDMPs) show a reduction in the number of opioid-related deaths \cite{pardo01} \cite{patrick01}. PDMPs reduce opioid deaths by tracking and monitoring batch-level opioid shipments to identify at risk prescribing and usage trends. Research also shows that states with legal medical marijuana dispensaries have lower opioid-related deaths \cite{pardo01}.


\section{An Overview of Pharmaceutical Supply Chains}
\subsection{Network Nodes}
Forward facing pharmaceutical supply chains include one or more of the following nodes including manufacturers, warehouses, distributors, and retailers. Reverse facing supply chain activities include collecting, recycling, redistributing, and disposing of unwanted medications.  
\subsubsection{Primary Manufacturing} Produces the main active ingredient \cite{Shah01}.
\subsubsection{Secondary Manufacturing} Often at a different geographic location for tax and labor reasons, secondary manufacturers combine the active ingredients produced by primary manufacturers and adding excipient substances. Secondary manufacturers produce distribution ready SKU medications through one or more of the following processes: granulation, compression, coating, quality control, and packaging \cite{Shah01}.
\subsubsection{Market Warehouses and Distribution Centers} Due to the cost of setup and cleaning, it is common for primary manufacturers to produce a years' worth of active ingredients for a particular medication in one batch. This strategy creates a lot of excess finished and work-in-progress inventory that is stored in warehouse and distribution centers \cite{Shah01}.
\subsubsection{Wholesalers} Roughly 80\% of demand flows through wholesalers. The industry is highly competitive and consolidated. The largest 5 wholesalers accounted for roughly 45\% of industry revenue \cite{Shah01}\cite{Hoovers01}.
\subsubsection{Pharmacies and Hospitals} The last node on the pharmaceutical supply chain before medications are distributed at a patient level. Major retailers include pharmacies CVS, Walgreens, Walmart, and Rite Aid and hospital systems such as Community Health Systems, Hospital Corporation of America, and Ascension Health. 
\subsubsection{Reverse Supply Chain} The reverse supply chain is often overlooked as a key component of the pharmaceutical supply chain network. A few people take their unwanted medications to the proper collection sites. Instead, medications are discarded in the trash and sewage. In fact, in 2003 \$560 million worth of pills were flushed down the toilet leading to dangerous compounds in sewage effluent, surface water, groundwater, and even drinking water \cite{Hua01}. Hua, Tang, and Wu \cite{Hua01} suggest a combination of government subsides, penalties, and marketing to encourage drug makers to collect unwanted and expired medications. 

\subsection{Weaknesses}
The nature of pharmaceutical products creates multiple weaknesses. 
\subsubsection{Lead Time} Lead times, the time it takes between manufacturing and end sale, can take up to 300 days \cite{Shah01}. As a result, high safety stocks are needed to react to future demand.
\subsubsection{High Service Levels} The necessity for on-time pharmaceutical products forces retailers to maintain high service levels, the targeted rate of stock-outs. In many cases and especially in hospitals, patient health relies on having the right medication at the right time. A failure to meet this immediate demand can have deadly consequences \cite{Kelle01}\cite{Hua01}.
\subsubsection{Imbalance of Information} Another major disadvantage is the lack of collaboration between raw material suppliers, manufacturers, warehouses, wholesalers, and retailers. ``The problem is that the different decision-makers do not have access to the same information regarding the state of the entire supply chain network, and in addition they usually operate under different objective functions'' \cite{Sahay01}. In this decentralized method, manufacturers have a difficult time forecasting demand. 
\subsubsection{Manufacturing Strategy} The mixture of manufacturers `push' strategy and retailers `pull' strategy, results in high safety stocks. At any given point, there is usually 4 to 24 weeks of finished goods in the supply chain  \cite{Shah01}. 
\subsubsection{Large Network} Medications past through several nodes before they are delivered to the market. Safety and security issues face agency conflicts as the capital cost to prevent theft and mismanagement is not equally spread across the supply chain. The number of nodes also increases the likelihood for counterfeits to enter the market. Between each node, medications are shipped and handled between multiple parties and often times across national and state borders.
\subsubsection{Government Regulation} The government heavily regulates pharmaceutical supply chains to ensure a safe and steady supply of medications. The Drug Quality and Security Act signed by President Barack Obama in 2013 introduced new regulations for manufacturing and distribution of pharmaceutical products. The policy mandates the creation of systems to trace lot-level transactions and systems to verify product legitimacy. In addition, any company within the supply chain must obtain federal licensure and authenticate the licensure of their trading partners. These required changes place immense financial pressure on pharmaceutical companies, drug distributors, and prescribers to develop sustainable supply chain solutions. The 2023 deadline gives pharmaceutical companies time to test and implement the most sustainable and practical solution \cite{opsis8}.

High inventory levels increase supply chain cost, the potential for theft, and the introduction of counterfeits. It is estimated that 10\% of the worldwide pharmaceuticals are counterfeit and approaching 25\% in developing countries \cite{Kelesidis01}. 

An imbalance of information between supply chain nodes increases cost and stock-outs. However, Nematollahi, Hosseini-Motlagh, and Heydari \cite{Nematollahi01} found that collaborative decision making through information sharing can increase economic benefits for the entire supply chain while also increasing drug fill rate.

Researchers at The Medicine Group propose using radio frequency identification (RFID) technology as a method to improve supply chain efficiency \cite{Taylor01}. RFID tracking tags are small microchips that are either printed, etched, stamped or vapor-deposited onto products and are intended to replace barcodes. RFID can be read without direct line of sight and at distances up to 30 feet. Research shows that RFID tags have the potential to reduce costs, increase transparency, and identify counterfeit lots. RFID tags have many advantages over current barcode tracking methods. RFID tags can hold up to 32,000 alphanumeric characters compared to just 20 in a barcode. RFID tags have a much higher upfront cost but decrease total supply chain cost due to the timely process to scan each individual barcode. And unlike RFID tags, barcodes are susceptible to wear and tear and are easily replicated. RFID technology has its flaws, each tag cost between 5-10 US cents, are vulnerable to electromagnetic interference and poor manufacturing, are larger, and require much a larger IT infrastructure \cite{Taylor01} \cite{opsis9}. From a security and transparency perspective, RFID technology is a good option to conform to The Drug Quality and Security Act.

\section{Blockchain in the Opioid Crisis}
New technologies are making pharmaceutical supply chains more efficient. President Trump's Commission on Combating Drug Addiction and the Opioid Crisis repeatedly mentions ``data sharing'' as a method to cope and limit the opioid crisis \cite{opsis3}.

Blockchain technology has the potential to reduce the opioid epidemic by strengthening a venerable supply chain while increasing data analysis on a batch level. Blockchains can provide a clear transnational history from raw material mining through patient level distribution. There are many advantages to using an open and decentralized supply chain that blockchain offers. Blockchain technology can help reduce the amount of counterfeit medications.  A more secure and transparent supply chain will identify at-risk batches and take 

ADD IN CONTENT FROM \cite{hitchingHealthcare}
\section{CONCLUSION}
Although still in its infancy, blockchain has the potential to be just as transformative as TCP/IP. Early and potential applications in healthcare and supply chain suggest that blockchain is indeed moving along the path of technology adoption. Because blockchain is a low-cost solution for supply chain management and provides security and transparency, it can be used for digital data and communication to overall the distribution of controlled substances such as opioids but this model has yet to be tested. But the computation and infrastructure cost for the entire model is low and should be tested to develop a proof of concept system that leverages blockchain to more securely distribute prescription opioids.

\begin{acks}

  The authors would like to thank Dr. Gregor von Laszewski for his support and suggestions to write this paper.

\end{acks}

\bibliographystyle{ACM-Reference-Format}
\bibliography{report} 

\appendix

We include an appendix with common issues that we see when students
submit papers. One particular important issue is not to use the
underscore in bibtex labels. Sharelatex allows this, but the
proceedings script we have does not allow this.

When you submit the paper you need to address each of the items in the
issues.tex file and verify that you have done them. Please do this
only at the end once you have finished writing the paper. To d this
cange TODO with DONE. However if you check something on with DONE, but
we find you actually have not executed it correcty, you will receive
point deductions. Thus it is important to do this correctly and not
just 5 minutes before the deadline. It is better to do a late
submission than doing the check in haste. 

\section{Issues}

\DONE{Example of done item: Once you fix an item, change TODO to DONE}

\subsection{Uncaught Bibliography Errors}

    \DONE{Bibtex labels cannot have any spaces, \_ or \& in it}
    \DONE{Citations in text showing as [?]: this means either your report.bib is not up-to-date or there is a spelling error in the label of the item you want to cite, either in report.bib or in report.tex}

\subsection{Formatting}

    \DONE{HID and i523 not included in the keywords}

\subsection{Writing Errors}

    \DONE{Errors in title capitalization}
    \DONE{A few spelling erros}

\subsection{Citation Issues and Plagiarism}

    \DONE{It is your responsibility to make sure no plagiarism occurs. The instructions and resources were given in the class}
    \DONE{Too few references}
    \DONE{The citation mark should not be in the beginning of the sentence or paragraph, but in the end, before the period mark. example: ... a library called Message Passing Interface(MPI) [7].}
    \DONE{Put a space between the citation mark and the previous word}

\subsection{Character Errors}

    \DONE{Pasting and copying from the Web often results in non-ASCII characters to be used in your text, please remove them and replace accordingly. This is the case for quotes, dashes and all the other special characters.}

\end{document}
